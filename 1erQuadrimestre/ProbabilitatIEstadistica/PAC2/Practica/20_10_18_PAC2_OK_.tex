% Options for packages loaded elsewhere
\PassOptionsToPackage{unicode}{hyperref}
\PassOptionsToPackage{hyphens}{url}
%
\documentclass[
]{article}
\usepackage{lmodern}
\usepackage{amssymb,amsmath}
\usepackage{ifxetex,ifluatex}
\ifnum 0\ifxetex 1\fi\ifluatex 1\fi=0 % if pdftex
  \usepackage[T1]{fontenc}
  \usepackage[utf8]{inputenc}
  \usepackage{textcomp} % provide euro and other symbols
\else % if luatex or xetex
  \usepackage{unicode-math}
  \defaultfontfeatures{Scale=MatchLowercase}
  \defaultfontfeatures[\rmfamily]{Ligatures=TeX,Scale=1}
\fi
% Use upquote if available, for straight quotes in verbatim environments
\IfFileExists{upquote.sty}{\usepackage{upquote}}{}
\IfFileExists{microtype.sty}{% use microtype if available
  \usepackage[]{microtype}
  \UseMicrotypeSet[protrusion]{basicmath} % disable protrusion for tt fonts
}{}
\makeatletter
\@ifundefined{KOMAClassName}{% if non-KOMA class
  \IfFileExists{parskip.sty}{%
    \usepackage{parskip}
  }{% else
    \setlength{\parindent}{0pt}
    \setlength{\parskip}{6pt plus 2pt minus 1pt}}
}{% if KOMA class
  \KOMAoptions{parskip=half}}
\makeatother
\usepackage{xcolor}
\IfFileExists{xurl.sty}{\usepackage{xurl}}{} % add URL line breaks if available
\IfFileExists{bookmark.sty}{\usepackage{bookmark}}{\usepackage{hyperref}}
\hypersetup{
  pdftitle={PAC2 (Pràctica)},
  pdfauthor={Josep Andreu Miralles},
  hidelinks,
  pdfcreator={LaTeX via pandoc}}
\urlstyle{same} % disable monospaced font for URLs
\usepackage[margin=1in]{geometry}
\usepackage{color}
\usepackage{fancyvrb}
\newcommand{\VerbBar}{|}
\newcommand{\VERB}{\Verb[commandchars=\\\{\}]}
\DefineVerbatimEnvironment{Highlighting}{Verbatim}{commandchars=\\\{\}}
% Add ',fontsize=\small' for more characters per line
\usepackage{framed}
\definecolor{shadecolor}{RGB}{248,248,248}
\newenvironment{Shaded}{\begin{snugshade}}{\end{snugshade}}
\newcommand{\AlertTok}[1]{\textcolor[rgb]{0.94,0.16,0.16}{#1}}
\newcommand{\AnnotationTok}[1]{\textcolor[rgb]{0.56,0.35,0.01}{\textbf{\textit{#1}}}}
\newcommand{\AttributeTok}[1]{\textcolor[rgb]{0.77,0.63,0.00}{#1}}
\newcommand{\BaseNTok}[1]{\textcolor[rgb]{0.00,0.00,0.81}{#1}}
\newcommand{\BuiltInTok}[1]{#1}
\newcommand{\CharTok}[1]{\textcolor[rgb]{0.31,0.60,0.02}{#1}}
\newcommand{\CommentTok}[1]{\textcolor[rgb]{0.56,0.35,0.01}{\textit{#1}}}
\newcommand{\CommentVarTok}[1]{\textcolor[rgb]{0.56,0.35,0.01}{\textbf{\textit{#1}}}}
\newcommand{\ConstantTok}[1]{\textcolor[rgb]{0.00,0.00,0.00}{#1}}
\newcommand{\ControlFlowTok}[1]{\textcolor[rgb]{0.13,0.29,0.53}{\textbf{#1}}}
\newcommand{\DataTypeTok}[1]{\textcolor[rgb]{0.13,0.29,0.53}{#1}}
\newcommand{\DecValTok}[1]{\textcolor[rgb]{0.00,0.00,0.81}{#1}}
\newcommand{\DocumentationTok}[1]{\textcolor[rgb]{0.56,0.35,0.01}{\textbf{\textit{#1}}}}
\newcommand{\ErrorTok}[1]{\textcolor[rgb]{0.64,0.00,0.00}{\textbf{#1}}}
\newcommand{\ExtensionTok}[1]{#1}
\newcommand{\FloatTok}[1]{\textcolor[rgb]{0.00,0.00,0.81}{#1}}
\newcommand{\FunctionTok}[1]{\textcolor[rgb]{0.00,0.00,0.00}{#1}}
\newcommand{\ImportTok}[1]{#1}
\newcommand{\InformationTok}[1]{\textcolor[rgb]{0.56,0.35,0.01}{\textbf{\textit{#1}}}}
\newcommand{\KeywordTok}[1]{\textcolor[rgb]{0.13,0.29,0.53}{\textbf{#1}}}
\newcommand{\NormalTok}[1]{#1}
\newcommand{\OperatorTok}[1]{\textcolor[rgb]{0.81,0.36,0.00}{\textbf{#1}}}
\newcommand{\OtherTok}[1]{\textcolor[rgb]{0.56,0.35,0.01}{#1}}
\newcommand{\PreprocessorTok}[1]{\textcolor[rgb]{0.56,0.35,0.01}{\textit{#1}}}
\newcommand{\RegionMarkerTok}[1]{#1}
\newcommand{\SpecialCharTok}[1]{\textcolor[rgb]{0.00,0.00,0.00}{#1}}
\newcommand{\SpecialStringTok}[1]{\textcolor[rgb]{0.31,0.60,0.02}{#1}}
\newcommand{\StringTok}[1]{\textcolor[rgb]{0.31,0.60,0.02}{#1}}
\newcommand{\VariableTok}[1]{\textcolor[rgb]{0.00,0.00,0.00}{#1}}
\newcommand{\VerbatimStringTok}[1]{\textcolor[rgb]{0.31,0.60,0.02}{#1}}
\newcommand{\WarningTok}[1]{\textcolor[rgb]{0.56,0.35,0.01}{\textbf{\textit{#1}}}}
\usepackage{graphicx,grffile}
\makeatletter
\def\maxwidth{\ifdim\Gin@nat@width>\linewidth\linewidth\else\Gin@nat@width\fi}
\def\maxheight{\ifdim\Gin@nat@height>\textheight\textheight\else\Gin@nat@height\fi}
\makeatother
% Scale images if necessary, so that they will not overflow the page
% margins by default, and it is still possible to overwrite the defaults
% using explicit options in \includegraphics[width, height, ...]{}
\setkeys{Gin}{width=\maxwidth,height=\maxheight,keepaspectratio}
% Set default figure placement to htbp
\makeatletter
\def\fps@figure{htbp}
\makeatother
\setlength{\emergencystretch}{3em} % prevent overfull lines
\providecommand{\tightlist}{%
  \setlength{\itemsep}{0pt}\setlength{\parskip}{0pt}}
\setcounter{secnumdepth}{-\maxdimen} % remove section numbering

\title{PAC2 (Pràctica)}
\author{Josep Andreu Miralles}
\date{15/10/2020}

\begin{document}
\maketitle

\hypertarget{exercici-1}{%
\section{EXERCICI 1}\label{exercici-1}}

Les dades de l'altura d'un grup d'homes reclutats per l'exèrcit a la fi
de segle XIX. (De AF Blakeslee, J. Hered. S: 551 (1914)) es podia
modelitzar amb una llei normal amb esperança 168 cm i desviació típica
6,75 cm. a) Calculeu la probabilitat que un home tingui una alçada entre
160 cm i 170 cm. b) Calculeu la probabilitat que un home mesuri
exactament 167,25 cm. c) Quins són els quartils de la distribució? d)
Per sobre de que alçada hauríem de trobar el 66\% dels homes?

\hypertarget{soluciuxf3}{%
\subsection{Solució}\label{soluciuxf3}}

\hypertarget{a-calculeu-la-probabilitat-que-un-home-tingui-una-aluxe7ada-entre-160-cm-i-170-cm.}{%
\subsection{a) Calculeu la probabilitat que un home tingui una alçada
entre 160 cm i 170
cm.}\label{a-calculeu-la-probabilitat-que-un-home-tingui-una-aluxe7ada-entre-160-cm-i-170-cm.}}

\begin{Shaded}
\begin{Highlighting}[]
\KeywordTok{pnorm}\NormalTok{ (}\DecValTok{170}\NormalTok{,}\DataTypeTok{mean=}\DecValTok{168}\NormalTok{,}\DataTypeTok{sd=}\FloatTok{6.75}\NormalTok{)}\OperatorTok{-}\KeywordTok{pnorm}\NormalTok{(}\DecValTok{160}\NormalTok{,}\DataTypeTok{mean=}\DecValTok{168}\NormalTok{,}\DataTypeTok{sd=}\FloatTok{6.75}\NormalTok{)}
\end{Highlighting}
\end{Shaded}

\begin{verbatim}
## [1] 0.498526
\end{verbatim}

\begin{Shaded}
\begin{Highlighting}[]
\NormalTok{prop_}\DecValTok{160}\NormalTok{_}\DecValTok{170}\NormalTok{<-}\StringTok{ }\KeywordTok{pnorm}\NormalTok{(}\KeywordTok{c}\NormalTok{(}\DecValTok{160}\NormalTok{,}\DecValTok{170}\NormalTok{),}\DataTypeTok{mean=}\DecValTok{168}\NormalTok{,}\DataTypeTok{sd=}\FloatTok{6.75}\NormalTok{,}\DataTypeTok{lower.tail=}\OtherTok{TRUE}\NormalTok{)}
\end{Highlighting}
\end{Shaded}

\hypertarget{b-calculeu-la-probabilitat-que-un-home-mesuri-exactament-16725-cm.}{%
\subsection{b) Calculeu la probabilitat que un home mesuri exactament
167,25
cm.}\label{b-calculeu-la-probabilitat-que-un-home-mesuri-exactament-16725-cm.}}

\begin{Shaded}
\begin{Highlighting}[]
\KeywordTok{dnorm}\NormalTok{(}\FloatTok{167.25}\NormalTok{,}\DecValTok{168}\NormalTok{,}\FloatTok{6.75}\NormalTok{)}
\end{Highlighting}
\end{Shaded}

\begin{verbatim}
## [1] 0.05873885
\end{verbatim}

\hypertarget{c-quins-suxf3n-els-quartils-de-la-distribuciuxf3}{%
\subsection{c) Quins són els quartils de la
distribució?}\label{c-quins-suxf3n-els-quartils-de-la-distribuciuxf3}}

\begin{Shaded}
\begin{Highlighting}[]
\KeywordTok{qnorm}\NormalTok{(}\FloatTok{0.25}\NormalTok{,}\DecValTok{168}\NormalTok{,}\FloatTok{6.75}\NormalTok{)}
\end{Highlighting}
\end{Shaded}

\begin{verbatim}
## [1] 163.4472
\end{verbatim}

\begin{Shaded}
\begin{Highlighting}[]
\KeywordTok{qnorm}\NormalTok{(}\FloatTok{0.5}\NormalTok{,}\DecValTok{168}\NormalTok{,}\FloatTok{6.75}\NormalTok{)}
\end{Highlighting}
\end{Shaded}

\begin{verbatim}
## [1] 168
\end{verbatim}

\begin{Shaded}
\begin{Highlighting}[]
\KeywordTok{qnorm}\NormalTok{(}\FloatTok{0.75}\NormalTok{,}\DecValTok{168}\NormalTok{,}\FloatTok{6.75}\NormalTok{)}
\end{Highlighting}
\end{Shaded}

\begin{verbatim}
## [1] 172.5528
\end{verbatim}

\hypertarget{d-per-sobre-de-que-aluxe7ada-hauruxedem-de-trobar-el-66-dels-homes}{%
\subsection{d) Per sobre de que alçada hauríem de trobar el 66\% dels
homes?}\label{d-per-sobre-de-que-aluxe7ada-hauruxedem-de-trobar-el-66-dels-homes}}

\begin{Shaded}
\begin{Highlighting}[]
\KeywordTok{qnorm}\NormalTok{(}\FloatTok{0.66}\NormalTok{,}\DecValTok{168}\NormalTok{,}\FloatTok{6.75}\NormalTok{,}\DataTypeTok{lower.tail=}\OtherTok{FALSE}\NormalTok{)}
\end{Highlighting}
\end{Shaded}

\begin{verbatim}
## [1] 165.2159
\end{verbatim}

\hypertarget{exercici-2}{%
\subsection{EXERCICI 2}\label{exercici-2}}

Càlcul de probabilitats amb R. Considerem Y una variable aleatòria que
segueix una distribució B(30;0.25). i Z una variable Poisson de
paràmetre 5. a) Calculeu, utilitzant R, P (Y = 10), P (Y \textless3), P
(Y\textgreater{} 15). b) Determineu a tal que P (X \textless= a) = 0.4.

\hypertarget{soluciuxf3-1}{%
\subsection{Solució}\label{soluciuxf3-1}}

\hypertarget{a-calculeu-utilitzant-r-p-y-10-p-y-3-p-y-15.}{%
\subsection{a) Calculeu, utilitzant R, P (Y = 10), P (Y \textless3), P
(Y\textgreater{}
15).}\label{a-calculeu-utilitzant-r-p-y-10-p-y-3-p-y-15.}}

\begin{Shaded}
\begin{Highlighting}[]
\KeywordTok{dbinom}\NormalTok{(}\DecValTok{10}\NormalTok{,}\DataTypeTok{size=}\DecValTok{30}\NormalTok{,}\DataTypeTok{prob=}\FloatTok{0.25}\NormalTok{)}
\end{Highlighting}
\end{Shaded}

\begin{verbatim}
## [1] 0.09086524
\end{verbatim}

\begin{Shaded}
\begin{Highlighting}[]
\KeywordTok{pbinom}\NormalTok{(}\DecValTok{2}\NormalTok{,}\DataTypeTok{size=}\DecValTok{30}\NormalTok{,}\DataTypeTok{prob=}\FloatTok{0.25}\NormalTok{)}
\end{Highlighting}
\end{Shaded}

\begin{verbatim}
## [1] 0.01059587
\end{verbatim}

\begin{Shaded}
\begin{Highlighting}[]
\KeywordTok{pbinom}\NormalTok{(}\DecValTok{16}\NormalTok{,}\DataTypeTok{size=}\DecValTok{30}\NormalTok{,}\DataTypeTok{prob=}\FloatTok{0.25}\NormalTok{,}\DataTypeTok{lower.tail=}\OtherTok{FALSE}\NormalTok{)}
\end{Highlighting}
\end{Shaded}

\begin{verbatim}
## [1] 0.0002156938
\end{verbatim}

\hypertarget{b-determineu-a-tal-que-p-x-a-0.4.}{%
\subsection{b) Determineu a tal que P (X \textless= a) =
0.4.}\label{b-determineu-a-tal-que-p-x-a-0.4.}}

\begin{Shaded}
\begin{Highlighting}[]
\KeywordTok{qpois}\NormalTok{(}\FloatTok{0.4}\NormalTok{,}\DataTypeTok{lambda=}\DecValTok{5}\NormalTok{)}
\end{Highlighting}
\end{Shaded}

\begin{verbatim}
## [1] 4
\end{verbatim}

\hypertarget{exercici-3}{%
\subsection{EXERCICI 3}\label{exercici-3}}

El fitxer ``vendes\_pac1\_P\_15\_2'' inclou les següents variables
referides a botigues: m2: superfície Ubi: ubicació (1 Centre ciutat, 2
Centre comercial, 3 Zona de vianants, 4 Barris, 5 Extraradi) PreuAm2:
preu del lloguer per m2 abans de fer reformes PreuDm2: preu del lloguer
per m2 després de fer reformes AugmentFact: Augment de la facturació
durant l'últim any. Hem generat la variable CatAM2 codificant la
variable PreuAm2 en quatre categoria, definides a partir dels quartils:
{[}min, Q\_1{]}, (Q1, Q2{]}, (Q2, Q3{]}, {[}Q3max{]}. Hem obtingut la
següent taula de contingència:

Si agafem una botiga a l'atzar, calculeu (manualment): a) La
probabilitat que la ubicació sigui ``Extraradi'' b) La probabilitat que
la ubicació sigui ``Extraradi'' i el preu del lloguer per metre quadrat
abans de la reforma estigui (Q1, Q2{]}. c) La probabilitat que la
ubicació sigui ``Extraradi'' sabent que el preu del lloguer per metre
quadrat abans de la reforma estigui (Q1, Q2{]}.

\hypertarget{soluciuxf3-2}{%
\subsection{Solució}\label{soluciuxf3-2}}

\hypertarget{a-la-probabilitat-que-la-ubicaciuxf3-sigui-extraradi}{%
\subsection{a) La probabilitat que la ubicació sigui
``Extraradi''}\label{a-la-probabilitat-que-la-ubicaciuxf3-sigui-extraradi}}

P(Extraradi) = P(Extraradi INTERSECCIO {[}8,9{]})+P(Extraradi
INTERSECCIO (9,12{]})+P(Extraradi INTERSECCIO (12,13{]})+P(Extraradi
INTERSECCIO (13,16{]})

P(Extraradi)=3/99+5/99=8/99=0.0808=8.08\%

\hypertarget{b-la-probabilitat-que-la-ubicaciuxf3-sigui-extraradi-i-el-preu-del-lloguer-per-metre-quadrat-abans-de-la-reforma-estigui-q1q2.}{%
\subsection{b) La probabilitat que la ubicació sigui ``Extraradi'' i el
preu del lloguer per metre quadrat abans de la reforma estigui
(Q1,Q2{]}.}\label{b-la-probabilitat-que-la-ubicaciuxf3-sigui-extraradi-i-el-preu-del-lloguer-per-metre-quadrat-abans-de-la-reforma-estigui-q1q2.}}

(Q1,Q2{]}=(9,12{]}

P(Extraradi INTERSECCIO (9,12{]}) = 5/99 = 0.0505 = 5.05\%

\hypertarget{c-la-probabilitat-que-la-ubicaciuxf3-sigui-extraradi-sabent-que-el-preu-del-lloguer-per-metre-quadrat-abans-de-la-reforma-estigui-q1q2.}{%
\subsection{c) La probabilitat que la ubicació sigui ``Extraradi''
sabent que el preu del lloguer per metre quadrat abans de la reforma
estigui
(Q1,Q2{]}.}\label{c-la-probabilitat-que-la-ubicaciuxf3-sigui-extraradi-sabent-que-el-preu-del-lloguer-per-metre-quadrat-abans-de-la-reforma-estigui-q1q2.}}

(Q1,Q2{]}=(9,12{]}

P(Extraradi CONDICIONAT A (9,12{]}) = P(Extraradi INTERSECCIO
(9,12{]})/P((9,12{]})

P(Extraradi CONDICIONAT A (9,12{]}) = (5/99)/(42/99) = 5/42 = 0.119 =
11.90\%

\end{document}
